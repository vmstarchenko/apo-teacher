\documentclass[12pt]{article}

\usepackage{amsthm}

\usepackage[T1,T2A]{fontenc}
\usepackage[utf8]{inputenc}
\usepackage[english,russian]{babel}

\usepackage[left=2cm,right=2cm,top=2cm,bottom=2cm]{geometry}

\usepackage{tabularx}
\usepackage{verbatim}
\usepackage{filecontents}
\usepackage{listings}
\usepackage[dvipsnames]{xcolor}
\usepackage{fancyvrb}

% custom
\newtheoremstyle{problemstyle}% name of the style to be used
{\topsep}   % measure of space to leave above the theorem. E.g.: 3pt
{\topsep}   % measure of space to leave below the theorem. E.g.: 3pt
{} % name of font to use in the body of the theorem
{1.5em}          % measure of space to indent
{\bfseries \Large} % name of head font
{\\}        % punctuation between head and body
{   }         % space after theorem head; " " = normal interword space
{\thmname{#1}\thmnumber{ №#2}\thmnote{: #3}
}

\theoremstyle{problemstyle}
\newtheorem{problem}{Задача}


% redefine \VerbatimInput
\RecustomVerbatimCommand{\VerbatimInput}{VerbatimInput}%
{fontsize=\footnotesize,
 %
 framesep=2em, % separation between frame and text
 rulecolor=\color{Gray},
 %
 %
 commandchars=\|\(\), % escape character and argument delimiters for
                      % commands within the verbatim
 commentchar=*        % comment character
}

\author{Старченко Владимир}
\title{Задачи курса школьного олимпиадного программирования.\\ Семинар 1.}


% ioexapmples
\newenvironment
{ioexamples}
{
  \newline \vspace{1em}
  {\bf \large  Пример входных и выходных данных} \vspace{0.5em} \\
  \tabularx{\textwidth}{|X|X|} \hline
  {\bf Ввод} & {\bf Вывод}  \\ \hline
}
{\endtabularx \vspace{0.5em}}

\newcommand{\example}[2]{ \verbatiminput{#1} & \verbatiminput{#2} \\ \hline }

\begin{document}


\maketitle

\begin{problem}[A + B]
  На стандартный ввод подаются два числа. Напечатать их сумму.
  \begin{ioexamples}
    \example{io/AplusB-0.in}{io/AplusB-0.out}
  \end{ioexamples}
\end{problem}


\begin{problem}[Even or Odd]
  На стандартный ввод подается число. Определить, является числи
  четным или нечетным. Если число четное, напечатать {\it even}.
  Если нечетное {\it odd}.
  \begin{ioexamples}
    \example{io/EvenOrOdd-0.in}{io/EvenOrOdd-0.out}
    \example{io/EvenOrOdd-1.in}{io/EvenOrOdd-1.out}
  \end{ioexamples}
\end{problem}


\begin{problem}[Factorial]
  На стандартный ввод подается число. Вывести факториал этого числа.
  \begin{ioexamples}
    \example{io/Factorial-0.in}{io/Factorial-0.out}
  \end{ioexamples}
\end{problem}


%% \begin{problem}[Factorial summ]
%%   На стандартный ввод подается число. Вывести сумму факториалов всех
%%   натуральных чисел меньше данного. Не разрешается делать болл
%%   \begin{ioexamples}
%%     \example{io/Factorial-0.in}{io/Factorial-0.out}
%%   \end{ioexamples}
%% \end{problem}

%% \begin{problem}[Function output generator]
%%   Необходимо написать функцию {\it generate\_values}, которая получает на вход
%%   аргументы {\it func}, {\it start}, {\it end}, {\it step}. Необходимо создать список, значения которого {\it func}, {\it start}, {\it end}, {\it step}
%%   \begin{ioexamples}
%%     \example{io/Factorial-0.in}{io/Factorial-0.out}
%%   \end{ioexamples}
%% \end{problem}


\end{document}
